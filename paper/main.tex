\documentclass{elsarticle}

\usepackage[utf8x]{inputenc}
\usepackage[table]{xcolor}
\usepackage{amsmath}
\usepackage{amssymb}
\usepackage{url}
\usepackage{graphicx}
\usepackage{color}
\usepackage{tabularx}
\usepackage{multirow}
\usepackage{float,lscape}
\usepackage{listings}
\usepackage{tkz-graph}
\lstset{language=C}
\rowcolors{3}{gray!10}{white}

\title{SAT Competition 2020\tnoteref{title}}
\tnotetext[title]{\url{satcompetition.github.io/2020}}

\author[jku]{Nils Froleyks}
\ead{nils.froleyks@jku.at}
\author[cmu]{Marijn Heule}
\ead{marijn@cmu.edu}
\author[kit]{Markus Iser}
\ead{markus.iser@kit.edu}
\author[hiit]{Matti Järvisalo}
\ead{matti.jarvisalo@helsinki.fi}
\author[ctu]{Martin Suda} 
\ead{martin.suda@cvut.cz}

\address[kit] {
KIT Department of Informatics\\
\url{markus.iser@kit.edu}\\[1em]
}

\newcommand{\todo}[1]{{\color{purple}Todo: #1}}

\begin{document}

\begin{abstract}
We describe the 2020 SAT Competition and provide a detailed analysis of its results.
\end{abstract}

\begin{keyword}
SAT, Competition, Retrospective
\end{keyword}

\maketitle

\section{Introduction}

\todo{Motivational paragraphs for SAT and the Competitions and what made this Competition Special}

\todo{Present some important pointers for SAT and CDCL}

\todo{Present the structure of the paper}


\section{Overview: Descriptions of Tracks, List of Participants}

\subsection{Main Track}

\subsection{Planning Track}

\subsection{Incremental Library Track}

\subsection{Parallel Track}

\subsection{Cloud Track}


\section{Selection of Benchmark Instances}

\subsection{Main Instances}

\subsection{Planning Instances}

\subsection{Incremental Library Applications and Instances}


\section{Competition Results}

\todo{for each track summarize the descriptions of the winning solvers}


\section{Analysis of Results}

\subsection{Essential Improvements in Winning Solvers}

\todo{Common and Specific Winning Strategies}

\subsection{Similarity of Solvers}

\todo{Rank Correlation}


\section{Conclusion}



\bibliographystyle{elsarticle-num}
\bibliography{main}

\end{document}
