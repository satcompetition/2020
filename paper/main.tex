\documentclass{elsarticle}

\usepackage[utf8x]{inputenc}
\usepackage[table]{xcolor}
\usepackage{amsmath}
\usepackage{amssymb}
\usepackage{url}
\usepackage{graphicx}
\usepackage{color}
\usepackage{tabularx}
\usepackage{multirow}
\usepackage{float,lscape}
\usepackage[ruled,linesnumbered,vlined]{algorithm2e}
\usepackage{tkz-graph}
\usepackage{lscape} 

\title{SAT Competition 2020\tnoteref{title}}
\tnotetext[title]{\url{satcompetition.github.io/2020}}

\author[jku]{Nils Froleyks}
\ead{nils.froleyks@jku.at}
\author[cmu]{Marijn Heule}
\ead{marijn@cmu.edu}
\author[kit]{Markus Iser}
\ead{markus.iser@kit.edu}
\author[hiit]{Matti Järvisalo}
\ead{matti.jarvisalo@helsinki.fi}
\author[ctu]{Martin Suda} 
\ead{martin.suda@cvut.cz}

\address[kit] {
KIT Department of Informatics\\
\url{markus.iser@kit.edu}\\[1em]
}

\newcommand{\todo}[1]{{\color{purple}Todo: #1}}

\begin{document}

\begin{abstract}
We describe the 2020 SAT Competition and provide a detailed analysis of its results.
\end{abstract}

\begin{keyword}
SAT, Competition
\end{keyword}

\maketitle

\section{Introduction}

\todo{Motivational paragraphs for SAT and the Competitions and what made this Competition Special}

\todo{Present some important pointers for SAT and CDCL}

\todo{Present the structure of the paper}


\section{Overview: Descriptions of Tracks, List of Participants}

\subsection{Main Tracks}

This year in the Main Track, we had a Planning Sub-Track. 
The general idea is, that next to a balanced selection of benchmark instances in main track, to each year also offer one domain-specific sub-track. 


%\newcommand{\stack}[2]{\begin{tabular}{@{}c@{}}#1 \\#2\end{tabular}}

% Stack a variable number of arguments:
\makeatletter
\newcommand{\stack}[1]{%
    \begin{tabular}{@{}l@{}}#1\checknextarg}
\newcommand{\checknextarg}{\@ifnextchar\bgroup{\gobblenextarg}{ \end{tabular}}}
\newcommand{\gobblenextarg}[1]{\\#1\@ifnextchar\bgroup{\gobblenextarg}{ \end{tabular}}}
\makeatother

\begin{table}[h]
\smaller
\begin{tabular}{|l|l|l|}
\hline
Team & Solver & Variant \\
\hline

\multirow{4}{*}{\stack{Biere, Fazekas, }{Fleury, Heisinger}} 
 & \multirow{3}{*}{Kissat-sc2020} & --\\
 &  & sat\\
 &  & unsat\\
\cline{2-3}
 & Cadical-sc2020 & --\\
\hline

\multirow{3}{*}{Zhang, Cai}
 & \multirow{3}{*}{MapleLCMDistChronoBT-DL} & Relaxed\\
 &  & Relaxed noTimeParam\\
 &  & Relaxed newTech\\
\hline

\multirow{2}{*}{\stack{Soos, Cai, Devriendt, }{Gocht, Shaw, Meel}}~
 & \multirow{2}{*}{CryptoMiniSat-CCAnr} & --\\
 &  & lsids\\
\hline

\stack{Soos, Selman, Kautz, }{Devriendt, Gocht}~ & CryptoMiniSat-WalkSAT & -- \\
\hline

\multirow{4}{*}{\stack{Hickey, Feng, }{Bacchus}}
 & \multirow{3}{*}{Cadical} & trail\\
 &  & alluip\\
 &  & alluip-trail\\
 \cline{2-3}
 & Maple-LCM-Dist & alluip-trail\\
\hline

\multirow{3}{*}{Kochemazov} & \multirow{2}{*}{MapleLCMDistChronoBT} & f2trc\\
 & & f2trc-s\\
 \cline{2-3}
 & MapleLCMDistChronoBT-DL & f2trc\\
\hline

\stack{Kochemazov, Zaikin, }{Kondratiev, Semenov} ~& MapleLCMDistChronoBT-DL-v3 & --\\
\hline

\multirow{4}{*}{\stack{Lonlac, }{Nguifo}}
 & \multirow{4}{*}{MapleLCMDiscChronoBT-DL-v3} & Undominated\\
 &  & Undominated Top16\\
 &  & Undominated Top24\\
 &  & Undominated Top36\\
\hline

\multirow{4}{*}{\stack{Tchinda, }{Djamegni}}
 & \multirow{4}{*}{ExMapleLCMDistChronoBT} & padc\_dl\\
 &  & padc\_dl\_ovau\_lin\\
 &  & padc\_dl\_ovau\_exp\\
 &  & psids\_dl\\
% \cline{2-3}& Glucose 3.0 & upGlucose-3.0\_PADC\\
\hline

Shaw, Meel & DurianSat & --\\
\hline

\multirow{2}{*}{Chen}
 & \multirow{2}{*}{MapleLCMDistChronoBT-DL} & Maple\_Mix\\
 &  & Maple\_Simp\\
\hline

Riveros & SLIME & --\\
\hline

\multirow{3}{*}{Li, Wu, Xu, Chen}
 & \multirow{3}{*}{MapleLCMDistChronoBT-DL} & Scavel\\
 &  & Scavel01\\
 &  & Scavel02\\
\hline

\multirow{2}{*}{\stack{Liang, Oh, Nejati, }{Poupart, Ganesh}}
 & \multirow{2}{*}{MapleCOMSPS\_LRB\_VSIDS\_2} & --\\
 &  & init\\
\hline

\multirow{4}{*}{\stack{Chowdhury, }{Müller, You}}
 & \multirow{4}{*}{MapleLCMDistChronoBT-DL-v2.2} & exp-V-LGB-MLD-CBT-DL \\
 &  & exp-V-L-MLDCBT-DL \\
 &  & exp-L-MLD-CBT-DL \\
 &  & exp-V-MLD-CBT-DL \\
\hline

\multirow{4}{*}{\stack{Li, Luo, Xiao, }{Li, Manyà, Lü}}
 & \multirow{4}{*}{MapleCM} & \texttt{+}dist \\
 &  & \texttt{+}dist\texttt{+}sattime2s\texttt{+}\texttt{-} \\
 &  & \texttt{+}dist\texttt{+}simp2\texttt{-}\texttt{-} \\
 &  & used\texttt{+}dist \\
\hline

Kaiser, Hartung & PauSat & -- \\
\hline

\multirow{2}{*}{Osama, Wijs} 
 & \multirow{2}{*}{ParaFROST} & -- \\
 &  & CBT \\
\hline
\end{tabular}
\caption{Teams and Solvers participating in the Main Tracks (Variant refers to a named configuration or a name of a hack). Note that many teams are only responsible for the variant of a previously existing base solver.}
\end{table}

\subsection{Incremental Library Track}

\begin{table}[h]
\begin{tabular}{|l|l|}
\hline
Chen & abcdsat-i20 \\
\hline
\stack{Biere, Fazekas, }{Fleury, Heisinger}~ & Cadical-sc2020 \\
\hline
\stack{Soos, Cai, }{Devriendt, Gocht, }{Shaw, Meel}~ & Cryptominisat-5 \\
\hline
Manthey & Riss-7.1.2\\
\hline
\end{tabular}
\caption{}
\end{table}

\subsection{Parallel Track}

\begin{table}[h]
\begin{tabular}{|l|l|}
\hline
\multirow{2}{*}{\stack{Vallade, Le Frioux, Baarir, }{Sopena, Kordon}}~
 & P-MCOMSPS-STR-32\\
 & P-MCOMSPS-STR-64\\
\hline
\multirow{2}{*}{\stack{Biere, Fazekas, }{Fleury, Heisinger}}
 & Plingeling\\
 & Treengeling\\
\hline
\multirow{2}{*}{Nabeshima, Inoue}
 & ManyGlucose-32\\
 & ManyGlucose-64\\
\hline
\multirow{2}{*}{Tchinda, Djamegni}
 & painlessmaplevone\\
 & painlessmaplevtwo\\
\hline
Li, Wu, Xu, Chen & syrupscavel\\
\hline
Chen & abcdsatptwenty\\
\hline
\end{tabular}
\caption{}
\end{table}

\subsection{Cloud Track}

\begin{table}[h]
\begin{tabular}{|l|l|}
\hline
Schreiber & mallob-mono\\
\hline
\stack{Ehlers, Kulczynski, }{Nowotka, Sieweck}~ & TopoSAT2\\
\hline
Riveros & Slime\\
\hline
\multirow{2}{*}{\stack{Biere, Fazekas, }{Fleury, Heisinger}}~ & Paracooba\\
& Paracooba-March\\
\hline
Hartung & CTSat\\
\hline
\end{tabular}
\end{table}


\section{Selection and Maintenance of Benchmark Instances}

GBD Tools is a new tool-chain for maintenance and distribuation of benchmark instances and instances features~\cite{Iser:2018:GBD}. 
Using the concept of instance identification via GBD Hash, which is the hash-sum of the unpacked and normalized DIMACS file, GBD Tools provide utilities\footnote{https://pypi.org/project/gbd-tools/} to query for instances with specific properties, while maintaining the association of instance-id and instance attributes in publicly available databases.\footnote{\url{https://gbd.iti.kit.edu}}

For selection of benchmarks in this competition, we used a database which contains information about instance author, family and result for a collection of instances used in all SAT competitive events dating back to 2006.\footnote{\url{https://gbd.iti.kit.edu/getdatabase/meta.db}} 
Like this it was possible to query for instances with desired properties, which gives the instance selection process a new quality. 
Using GBD Server, each instance also has an URL, such that instances can also be distributed by using their identifier: \url{https://gbd.iti.kit.edu/file/<gbd-hash>}.


\subsection{Selection of Instances}

The ``Bring Your Own Benchmarks'' (BYOB) rule (applied as of SAT Competition 2017~\cite{SC2017}) requires solver authors to submit $20$ new benchmark instances in order to participate in the competition. At least $10$ of these instances are required to be ``interesting'', whereby interesting is defined as Minisat~\cite{Niklas:2003:Minisat} needs at least one minute to solve it and the authors own solver does not run into a timeout for the instance. 

In total, $28$ authors followed our calls for participation and benchmark instances, thus contributing to a set of $1260$ previously unseen benchmark instances with a variety of $27$ different instance families. 
By filtering out instances with could be solved by Minisat in less than $10$ minutes, we obtained an initial set of $1012$ instances. 

In order to compile a balanced set of $300$ benchmark instances, we used the procedure which is depicted in Algorithm~\ref{algo:select} to select a maximium of $14$ submissions per author. 
Per author, if possible, we first randomly selected $7$ satisfiable and $7$ unsatisfiable instances (lines~3 and~4). 
If this did not yield a total of $14$ instances, we added instances of yet unknown outcome (lines~5-7). 
Of the such obtained $308$ instances, we randomly removed $8$ satisfiable instances, yielding a total of $114$ satisfiable, $78$ unsatisfiable and $108$ instances of unknown result. 

\begin{algorithm}[t]
\DontPrintSemicolon

\KwData{$I$ : Set of Instances, $A$ : Set of Authors}
\KwData{Functions $\alpha : I \rightarrow A$ and $\sigma : I \rightarrow \{\mathsf{sat}, \mathsf{unsat}, \mathsf{unknown}\}$}
\KwResult{$S$ : Set of Selected Instances}
\SetKwFunction{rand}{$\mathsf{random.choice}$}
\BlankLine
$S \leftarrow \emptyset$\;

\For {$a \in A$} {
	$I_a^+ \leftarrow$ \rand{$\{ e \in I \mid \alpha(e) = a \land \sigma(e) = \mathsf{sat} \}$, $k=7$}\;	
	$I_a^- \leftarrow$ \rand{$\{ e \in I \mid \alpha(e) = a \land \sigma(e) = \mathsf{unsat} \}$, $k=7$}\;	
	\If {$|I_a^+|+|I_a^-| < 14$}{
		$l \leftarrow 14 - (|I_a^+|+|I_a^-|)$\;
		$I_a^? \leftarrow$ \rand{$\{ e \in I \mid \alpha(e) = a \land \sigma(e) = \mathsf{unknown} \}$, $k=l$}\;
	}
	$S \leftarrow S \cup I_a^+ \cup I_a^- \cup I_a^?$\;	
}
\Return $S$\;

\caption{Benchmark Instance Selection}
\label{algo:select}
\end{algorithm}

To obtain the final compilation of $400$ benchmark instances, we augmented this set with $100$ instances which have been used in previous competitions. 
We randomly selected $21$ satisfiable, $57$ unsatisfiable and $22$ unknown instances to yield a total of $135$ satisfiable, $135$ unsatisfiable and $130$ instances of unkown result. 
Furthermore, we made sure not to select instances of a family which is already represented in the set of newly submitted instances and excluded random, agile and planning instances (due to the planning track). 


\subsection{Planning Instances}

\subsection{Incremental Library Applications and Instances}


\section{Competition Results}

\todo{for each track summarize the descriptions of the winning solvers}


\section{Analysis of Results}

\subsection{Essential Improvements in Winning Solvers}

\todo{Common and Specific Winning Strategies}

\subsection{Similarity of Solvers}

\todo{Rank Correlation}


\section{Conclusion}



\bibliographystyle{elsarticle-num}
\bibliography{main}

\end{document}
